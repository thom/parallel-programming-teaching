%==============================================================================
% presentation.tex
%==============================================================================


%==============================================================================
% Configuration
%==============================================================================

% Internationalisation
\usepackage[utf8]{inputenc}
\usepackage[T1]{fontenc}
% \usepackage[ngerman]{babel}

% Different packages
\usepackage{url}
\usepackage{color,listings,paralist}
\usepackage{enumerate}
\usepackage{tabularx}
\usepackage{alltt}

% Use default Acrobat reader fonts
\usepackage{mathpazo}

% Use CM fonts (increases document size)
\usepackage{ae}

% Use images
\usepackage{graphicx}

% Configure beamer
\usetheme[secheader]{Ikhono}
\usefonttheme[onlylarge]{structurebold}
\setbeamertemplate{navigation symbols}{}

% Variables
\providecommand{\Title}{Parallel Programming}
\providecommand{\Subtitle}{Recitation Session 8}
\providecommand{\Author}{Thomas Weibel <weibelt@ethz.ch>}
\providecommand{\Institute}{Laboratory for Software Technology, \\
  Swiss Federal Institute of Technology Z\"urich}
\providecommand{\Date}{April 29, 2010}

% PDF settings
\hypersetup{
  pdftitle={\Title, \Subtitle},
  pdfauthor={\Author},
  pdfsubject={\Institute},
  pdfkeywords={parallel programming} 
}

% Titlepage
\title{\Title}
\subtitle{\Subtitle}
\author{\Author}
\institute{\Institute}
\date{\Date}

% Listings
\lstdefinestyle{Default}{
  language=Java,
  tabsize=2,
  mathescape=true,
  inputencoding=utf8,
  showstringspaces=false,
  fontadjust=true,
  basicstyle=\ttfamily,
  keywordstyle=\color{blue}\bfseries,
}
\lstset{style=Default}


%==============================================================================
% Document
%==============================================================================

\begin{document}


% Titlepage
\begin{frame}[plain]
  \titlepage
\end{frame}


\section*{Introduction}

\begin{frame}{Executive Summary}
  \begin{itemize}
  \item TODO
  \end{itemize}
\end{frame}


\section{Read/Write Lock}

\begin{frame}{Outline}
  \tableofcontents[current]
\end{frame}

\begin{frame}{Read/Write Lock}
  \begin{itemize}
  \item Many shared objects have the property that most method calls
    return information about the object's state without modifying the
    object ({\bf readers}) while only a small number of calls actually
    modify the object ({\bf writers})
  \item There is no need for readers to synchronize with one another
    \begin{itemize}
    \item It is perfectly safe for them to access the object
      concurrently
    \end{itemize}
  \item Writers, on the other hand, must lock out readers as well as
    other writers
  \item A Read/Write Lock allows multiple readers or a single
    writer to enter the critical section concurrently
  \end{itemize}
\end{frame}

\begin{frame}{Assignment 7}
  \begin{itemize}
  \item Implement a Read/Write Lock
  \item At most four threads
  \item At most two reader threads (shared access is allowed) and one
    writer thread
  \item A thread that executes \lstinline!read()! is a reader
    \begin{itemize}
    \item At a later time it can be a writer...
    \end{itemize}
  \end{itemize}
\end{frame}

\begin{frame}[fragile]{\lstinline!Monitor!}
\begin{lstlisting}[basicstyle=\fontsize{10}{12}\selectfont\ttfamily]
public class Monitor {
  final int MAX_THREADS;
  final int MAX_READERS = 2;
  FIFOQueue waitList;
  int readers = 0;
  int writers = 0;
  boolean writing = false;
  
  public Monitor(int maxThreads) {
    MAX_THREADS = maxThreads;
    waitList = new FIFOQueue(maxThreads);
  }  
  
  public void readLock()    { /* ... */ }
  public void readUnlock()  { /* ... */ }
  public void writeLock()   { /* ... */ }
  public void writeUnlock() { /* ... */ }
}
\end{lstlisting}
\end{frame}

\begin{frame}[fragile]{\lstinline!readLock()!}
\begin{lstlisting}[basicstyle=\fontsize{7}{9}\selectfont\ttfamily]
public synchronized void readLock() {
  if (readers >= MAX_READERS || writing || !waitList.isEmpty()) {
    waitList.enq(Thread.currentThread().getId());

    while (true) {
      try {
        wait();
      } catch (InterruptedException e) {
        e.printStackTrace();
      }

      if (waitList.getFirstItem() == Thread.currentThread().getId()
          && !writing && readers < MAX_READERS) {
        waitList.deq();
        break;
      }
    }
  }

  readers++;
  if (readers < MAX_READERS)
    notifyAll();
  System.out.println("READ LOCK ACQUIRED " + readers);
}
\end{lstlisting}
\end{frame}

\begin{frame}[fragile]{\lstinline!readUnlock()!}
\begin{lstlisting}
public synchronized void readUnlock() {
  readers--;
  System.out.println("READ LOCK RELEASED " + 
                     readers);
  notifyAll();
}
\end{lstlisting}
\end{frame}

\begin{frame}[fragile]{\lstinline!writeLock()!}
\begin{lstlisting}[basicstyle=\fontsize{7}{9}\selectfont\ttfamily]
public synchronized void writeLock() {
  if (readers > 0 || writers > 0 || !waitList.isEmpty()) {
    waitList.enq(Thread.currentThread().getId());
    while (true) {
      try {
        wait();
      } catch (InterruptedException e) {
        System.out.println(e.getMessage());
      }

      if (waitList.getFirstItem() == Thread.currentThread().getId()
          && !writing && readers == 0) {
        waitList.deq();
        break;
      }
    }
  }
  
  writers++;
  writing = true;
  System.out.println("WRITE LOCK ACQUIRED " + writers);
}
\end{lstlisting}
\end{frame}

\begin{frame}[fragile]{\lstinline!writeUnlock()!}
\begin{lstlisting}
public synchronized void writeUnlock() {
  writing = false;
  writers--;    
  System.out.println("WRITE LOCK RELEASED " + 
                     writers);
  notifyAll();
}
\end{lstlisting}
\end{frame}


\section{TODO}

\begin{frame}{Outline}
  \tableofcontents[current]
\end{frame}

\begin{frame}{TODO}
  \begin{itemize}
  \item TODO
  \end{itemize}
\end{frame}


\section{TODO}

\begin{frame}{Outline}
  \tableofcontents[current]
\end{frame}

\begin{frame}{TODO}
  \begin{itemize}
  \item TODO
  \end{itemize}
\end{frame}


\section{TODO}

\begin{frame}{Outline}
  \tableofcontents[current]
\end{frame}

\begin{frame}{TODO}
  \begin{itemize}
  \item TODO
  \end{itemize}
\end{frame}


\section*{Outro}

\begin{frame}{Summary}
  \begin{itemize}
  \item TODO
\end{itemize}
\end{frame}

\end{document}
