%==============================================================================
% presentation.tex
%==============================================================================


%==============================================================================
% Configuration
%==============================================================================

% Internationalisation
\usepackage[utf8]{inputenc}
\usepackage[T1]{fontenc}
% \usepackage[ngerman]{babel}

% Different packages
\usepackage{url}
\usepackage{color,listings,paralist}
\usepackage{enumerate}
\usepackage{tabularx}
\usepackage{alltt}

% Use default Acrobat reader fonts
\usepackage{mathpazo}

% Use CM fonts (increases document size)
\usepackage{ae}

% Use images
\usepackage{graphicx}

% Configure beamer
\usetheme[secheader]{Ikhono}
\usefonttheme[onlylarge]{structurebold}
\setbeamertemplate{navigation symbols}{}

% Variables
\providecommand{\Title}{Parallel Programming}
\providecommand{\Subtitle}{Recitation Session 8}
\providecommand{\Author}{Thomas Weibel <weibelt@ethz.ch>}
\providecommand{\Institute}{Laboratory for Software Technology, \\
  Swiss Federal Institute of Technology Z\"urich}
\providecommand{\Date}{April 29, 2010}

% PDF settings
\hypersetup{
  pdftitle={\Title, \Subtitle},
  pdfauthor={\Author},
  pdfsubject={\Institute},
  pdfkeywords={parallel programming} 
}

% Titlepage
\title{\Title}
\subtitle{\Subtitle}
\author{\Author}
\institute{\Institute}
\date{\Date}

% Listings
\lstdefinestyle{Default}{
  language=Java,
  tabsize=2,
  mathescape=true,
  inputencoding=utf8,
  showstringspaces=false,
  fontadjust=true,
  basicstyle=\ttfamily,
  keywordstyle=\color{blue}\bfseries,
}
\lstset{style=Default}


%==============================================================================
% Document
%==============================================================================

\begin{document}


% Titlepage
\begin{frame}[plain]
  \titlepage
\end{frame}


\section*{Introduction}

\begin{frame}{Executive Summary}
  \begin{itemize}
  \item TODO
  \end{itemize}
\end{frame}


\section{Mutual Exclusion Proofs}

\begin{frame}{Outline}
  \tableofcontents[current]
\end{frame}

\begin{frame}[fragile]{Notation}
\begin{lstlisting}
public void run() {
  while (true) {
    mysignal.request();
    while (true) {
      if (othersignal.read() == 1) break;
      mysignal.free();
      mysignal.request();
    }
    // critical section
    mysignal.free();
  }
}
\end{lstlisting}
\end{frame}

\begin{frame}[fragile]{Notation}
  \begin{columns}[c]
    \begin{column}{0.50\textwidth}
\begin{lstlisting}[basicstyle=\fontsize{9}{11}\selectfont\ttfamily]
A1 // non-critical section
A2 turn0.flag = 0;
A3 while(true)
     if(turn1.flag == 1) 
       break;
A4   turn0.flag = 1;
A5   turn0.flag = 0;
   }
A6 // critical section
A7 turn0.flag = 1;
\end{lstlisting}
    \end{column}
    \begin{column}{0.50\textwidth}
\begin{lstlisting}[basicstyle=\fontsize{9}{11}\selectfont\ttfamily]
B1 // non-critical section
B2 turn1.flag = 0;
B3 while(true) {
     if(turn0.flag == 1) 
       break;
B4   turn1.flag = 1;
B5   turn1.flag = 0;
   }
B6 // critical section
B7 turn1.flag = 1;
\end{lstlisting}
    \end{column}
  \end{columns}
\end{frame}

\begin{frame}[fragile]{Implication}
  \begin{enumerate}
  \item \lstinline!at(A6)! $\rightarrow$ \lstinline!turn0.flag == 0!
  \end{enumerate}

  \vspace{\stretch{1}}

\begin{lstlisting}
  A1 // non-critical section
  A2 turn0.flag = 0;
  A3 while(true)
       if(turn1.flag == 1) 
         break;
  A4   turn0.flag = 1;
  A5   turn0.flag = 0;
     }
  A6 // critical section
  A7 turn0.flag = 1;
\end{lstlisting}

  \vspace{\stretch{1}}

  Why is invariant (1) true \lstinline!at(A1)!, \lstinline!at(A2)!,
  \lstinline!at(A3)!, \lstinline!at(A4)!, \lstinline!at(A5)!,
  \lstinline!at(A7)!?
\end{frame}

\begin{frame}{Implication: Truth Table}
  $$A \rightarrow B$$

  \vspace{\stretch{1}}

  \begin{center}
    \begin{tabular}{|c|c c|}
      \hline
      $\rightarrow$ & 1 & 0 \\\hline
      1 & 1 & 0 \\
      0 & 1 & 1 \\\hline
    \end{tabular}
  \end{center}

  \vspace{\stretch{1}}

  \begin{enumerate}
  \item \lstinline!at(A6)! $\rightarrow$ \lstinline!turn0.flag == 0!
  \end{enumerate}

  \vspace{\stretch{1}}

  \begin{itemize}
  \item \lstinline!at(A1)!: $0 \rightarrow x = 1$
  \item \lstinline!at(A2)!: $0 \rightarrow x = 1$
  \item \lstinline!at(A3)!: $0 \rightarrow x = 1$
  \item \lstinline!at(A4)!: $0 \rightarrow x = 1$
  \item \lstinline!at(A5)!: $0 \rightarrow x = 1$
  \item \lstinline!at(A7)!: $0 \rightarrow x = 1$
  \end{itemize}
\end{frame}

\begin{frame}[fragile]{Equivalence}
  \begin{enumerate}
  \item \lstinline!turn0.flag == 0! $\leftrightarrow$ 
    \lstinline!(at(A3) $\vee$ at(A4) $\vee$ at(A6) $\vee$ at(A7))!
  \end{enumerate}

  \vspace{\stretch{1}}

\begin{lstlisting}
  A1 // non-critical section
  A2 turn0.flag = 0;
  A3 while(true)
       if(turn1.flag == 1) 
         break;
  A4   turn0.flag = 1;
  A5   turn0.flag = 0;
     }
  A6 // critical section
  A7 turn0.flag = 1;
\end{lstlisting}

  \vspace{\stretch{1}}

  \lstinline!A4 $\rightarrow$ A5! and \lstinline!A7 $\rightarrow$ A1!:
  \lstinline!S0.flag == 1!, why does invariant (1) still hold?
\end{frame}

\begin{frame}{Equivalence: Truth Table}
  $$A \leftrightarrow B$$

  \vspace{\stretch{1}}

  \begin{center}
    \begin{tabular}{|c|c c|}
      \hline
      $\leftrightarrow$ & 1 & 0 \\\hline
      1 & 1 & 0 \\
      0 & 0 & 1 \\\hline
    \end{tabular}
  \end{center}

  \vspace{\stretch{1}}

  \begin{enumerate}
  \item \lstinline!turn0.flag == 0! $\leftrightarrow$ 
    \lstinline!(at(A3) $\vee$ at(A4) $\vee$ at(A6) $\vee$ at(A7))!
  \end{enumerate}

  \vspace{\stretch{1}}

  \begin{itemize}  
  \item \lstinline!A4 $\rightarrow$ A5! $==$ \lstinline!at(A5)!: $0
    \leftrightarrow 0 = 1$
  \item \lstinline!A7 $\rightarrow$ A1! $==$ \lstinline!at(A1)!: $0
    \leftrightarrow 0 = 1$
  \end{itemize}
\end{frame}


\section{Read/Write Lock}

\begin{frame}{Outline}
  \tableofcontents[current]
\end{frame}

\begin{frame}{Read/Write Lock}
  \begin{itemize}
  \item Many shared objects have the property that most method calls
    return information about the object's state without modifying the
    object ({\bf readers}) while only a small number of calls actually
    modify the object ({\bf writers})
  \item There is no need for readers to synchronize with one another
    \begin{itemize}
    \item It is perfectly safe for them to access the object
      concurrently
    \end{itemize}
  \item Writers, on the other hand, must lock out readers as well as
    other writers
  \item A Read/Write Lock allows multiple readers or a single
    writer to enter the critical section concurrently
  \end{itemize}
\end{frame}

\begin{frame}{Assignment 7}
  \begin{itemize}
  \item Implement a Read/Write Lock
  \item At most four threads
  \item At most two reader threads (shared access is allowed) and one
    writer thread
  \item A thread that executes \lstinline!read()! is a reader
    \begin{itemize}
    \item At a later time it can be a writer...
    \end{itemize}
  \end{itemize}
\end{frame}

\begin{frame}[fragile]{\lstinline!Monitor!}
\begin{lstlisting}[basicstyle=\fontsize{10}{12}\selectfont\ttfamily]
public class Monitor {
  final int MAX_THREADS;
  final int MAX_READERS = 2;
  FIFOQueue waitList;
  int readers = 0;
  int writers = 0;
  boolean writing = false;
  
  public Monitor(int maxThreads) {
    MAX_THREADS = maxThreads;
    waitList = new FIFOQueue(maxThreads);
  }  
  
  public void readLock()    { /* ... */ }
  public void readUnlock()  { /* ... */ }
  public void writeLock()   { /* ... */ }
  public void writeUnlock() { /* ... */ }
}
\end{lstlisting}
\end{frame}

\begin{frame}[fragile]{\lstinline!readLock()!}
\begin{lstlisting}[basicstyle=\fontsize{7}{9}\selectfont\ttfamily]
public synchronized void readLock() {
  if (readers >= MAX_READERS || writing || !waitList.isEmpty()) {
    waitList.enq(Thread.currentThread().getId());

    while (true) {
      try {
        wait();
      } catch (InterruptedException e) {
        e.printStackTrace();
      }

      if (waitList.getFirstItem() == Thread.currentThread().getId()
          && !writing && readers < MAX_READERS) {
        waitList.deq();
        break;
      }
    }
  }

  readers++;
  if (readers < MAX_READERS)
    notifyAll();
  System.out.println("READ LOCK ACQUIRED " + readers);
}
\end{lstlisting}
\end{frame}

\begin{frame}[fragile]{\lstinline!readUnlock()!}
\begin{lstlisting}
public synchronized void readUnlock() {
  readers--;
  System.out.println("READ LOCK RELEASED " + 
                     readers);
  notifyAll();
}
\end{lstlisting}
\end{frame}

\begin{frame}[fragile]{\lstinline!writeLock()!}
\begin{lstlisting}[basicstyle=\fontsize{7}{9}\selectfont\ttfamily]
public synchronized void writeLock() {
  if (readers > 0 || writers > 0 || !waitList.isEmpty()) {
    waitList.enq(Thread.currentThread().getId());
    while (true) {
      try {
        wait();
      } catch (InterruptedException e) {
        System.out.println(e.getMessage());
      }

      if (waitList.getFirstItem() == Thread.currentThread().getId()
          && !writing && readers == 0) {
        waitList.deq();
        break;
      }
    }
  }
  
  writers++;
  writing = true;
  System.out.println("WRITE LOCK ACQUIRED " + writers);
}
\end{lstlisting}
\end{frame}

\begin{frame}[fragile]{\lstinline!writeUnlock()!}
\begin{lstlisting}
public synchronized void writeUnlock() {
  writing = false;
  writers--;    
  System.out.println("WRITE LOCK RELEASED " + 
                     writers);
  notifyAll();
}
\end{lstlisting}
\end{frame}


\section{Equivalence of Semaphores and Monitors}

\begin{frame}{Outline}
  \tableofcontents[current]
\end{frame}

\begin{frame}{Classroom Exercise}
  \begin{itemize}
  \item Are semaphores and monitors equivalent?
  \item How can you implement semaphores with monitors?
  \item How can you implement monitors with semaphores?
    \begin{itemize}
    \item What about \lstinline!wait()! and \lstinline!notifyAll()!?
    \end{itemize}
  \end{itemize}
\end{frame}

\begin{frame}{Semaphores and Monitors}
  \begin{itemize}
  \item Monitor: model for synchronized methods in Java
  \item Both constructs are equivalent
  \item \alert{One can be used to implement the other}
  \end{itemize}
\end{frame}

\begin{frame}[fragile]{Semaphore Implementation}
\begin{lstlisting}
public class Semaphore {
  private int value;
  public Semaphore() { 
    value = 0; 
  }
  public Semaphore(int k) { 
    value = k; 
  }
  public synchronized void acquire() { 
    /* see later */ 
  }
  public synchronized void release() { 
    /* see later */ 
  }   
}
\end{lstlisting}
\end{frame}

\begin{frame}[fragile]{Semaphore Implementation: \lstinline!acquire()!}
\begin{lstlisting}
public synchronized void acquire() {
  while (value == 0) {
    try {
      wait();
    }
    catch (InterruptedException e) { 
    }
  }
  value--;
}
\end{lstlisting}
\end{frame}

\begin{frame}[fragile]{Semaphore Implementation: \lstinline!release()!}
\begin{lstlisting}
public synchronized void release() {
  ++value;
  notifyAll();
}
\end{lstlisting}
\end{frame}

\begin{frame}{Monitor with Semaphores}
  We need 2 semaphores:
  \begin{itemize}
  \item	One to make sure that only one synchronized method executes at any given time
    \begin{itemize}
    \item call this the ``access semaphore'' \lstinline!access!
    \item binary semaphore
    \end{itemize}
  \item One semaphore to line up threads that are waiting for some condition
    \begin{itemize}
    \item call this the ``condition semaphore'' \lstinline!cond!
    \item counting (general) semaphore
    \item threads that wait must do an ``acquire''
    \end{itemize}
  \end{itemize}

  \vspace{\stretch{1}}

  \begin{block}{For convenience}
    Counter \lstinline!waitThread! to count number of waiting threads
    i.e., threads in queue for \lstinline!cond!
  \end{block}  
\end{frame}

\begin{frame}{Monitor with Semaphores}
  \begin{enumerate}
  \item Frame all synchronized methods with \lstinline!access.acquire()! and
    \lstinline!access.release()!
    \begin{itemize}
    \item This ensures that only one thread executes a synchronized
      method at any point in time
    \item Recall: \lstinline!access! is binary.
    \end{itemize}
  \item Translate \lstinline!wait()! and \lstinline!notifyAll()! to give threads waiting in
    line a chance to progress (these threads use \lstinline!cond!)
  \end{enumerate}  
\end{frame}

\begin{frame}[fragile]{Monitor with Semaphores: Auxiliary Fields}
\begin{lstlisting}
class FooBar {
  private Semaphore access;
  private Semaphore cond;
  private int waitThread = 0;

  public FooBar() { 
    access = new Semaphore(1);
    cond = new Semaphore(0);
  }
  
  // continued
}
\end{lstlisting}  
\end{frame}

\begin{frame}[fragile]{(1) Framing all methods}
\begin{lstlisting}
  public void qux() {
    // Ensure mutual exclusion
    access.acquire(); 

    // Critical section

    access.release();
  }
\end{lstlisting}

  \vspace{\stretch{1}}

  is equivalent to

  \vspace{\stretch{1}}

\begin{lstlisting}
  public synchronized void qux() {
    // Critical section
  }
\end{lstlisting}
\end{frame}

\begin{frame}[fragile]{(2) Translate \lstinline!wait()!}
\begin{lstlisting}
waitThread++;

// other threads can execute 
// synchronized methods 
access.release(); 

// wait till condition changes
cond.acquire(); 
access.acquire();

waitThread--;
\end{lstlisting}
\end{frame}

\begin{frame}[fragile]{(2) Translate \lstinline!notifyAll()!}
\begin{lstlisting}
if (waitThread > 0) {
  for (int i=0; i < waitThread; i++) { 
    cond.release();
  }
}	
\end{lstlisting}

  \vspace{\stretch{1}}

  \begin{itemize}
  \item All threads waiting are released and will compete to
    (re)acquire \lstinline!access!
  \item They decrement \lstinline!waitThread! after they leave
    \lstinline!cond.acquire()!
  \item Note that to enter the line (i.e., increment
    \lstinline!waitThread!) the thread must hold the access semaphore
    \lstinline!access!
  \end{itemize}
\end{frame}

\begin{frame}{(2) Translate \lstinline!wait()! and \lstinline!notifyAll()!}
  \begin{itemize}
  \item Recall that \lstinline!access.release()! is done at the end of
    the synchronized method
  \item So all the threads that had lined up waiting for
    \lstinline!cond! compete to get access to \lstinline!access!
  \item No thread can line up while the \lstinline!cond.release()!
    operations are done since this thread holds \lstinline!access!
  \end{itemize}
\end{frame}

\begin{frame}{Note}
  \begin{itemize}
  \item We wake up all threads -- they might not be able to enter
    their critical section if the condition they waited for does not
    hold, but all threads get a chance.
  \item \lstinline!notifyAll()! calls \lstinline!cond.release()!
    \lstinline!waitThread!-times
    \begin{itemize}
    \item If 3 threads were waiting, \lstinline!cond.value! is set to 3.
    \item Now one of thread wakes up, decrements
      \lstinline!waitThreads!, runs to the end and again calls
      \lstinline!cond.release()! 2 times $\rightarrow$
      \lstinline!cond.value! will now be at 4 even though only 2
      threads are in the wait section of the code
    \item A thread may awake a few times and not find that the
      condition has changed. As long as all the \lstinline!wait()! are
      in \lstinline!while (some_condition)! loop there will be no harm
    \end{itemize}
  \end{itemize}
\end{frame}

\begin{frame}[fragile]{\lstinline!wait()! in while-loop}
\begin{lstlisting}[basicstyle=\fontsize{9}{11}\selectfont\ttfamily]
public void insert(Object o) 
  throws InterruptedException {
  access.acquire();
  while (isFull()) {
    waitThread++;
    access.release(); // let other thread access object
    cond.acquire(); // wait for change of state
    access.acquire()
    waitThread--;
  }
  doInsert(o);	
  if (waitThread > 0) { 
    for (int i; i < waitThread; i++) {
      cond.release(); 
    } 
  }
  access.release(); 
}
\end{lstlisting}
\end{frame}


\section{Lock Proof}

\begin{frame}{Outline}
  \tableofcontents[current]
\end{frame}

\begin{frame}[fragile]{Classroom Exercise}
  \begin{columns}[c]
    \begin{column}{0.50\textwidth}
\begin{lstlisting}[basicstyle=\fontsize{9}{11}\selectfont\ttfamily]
class MyLock implements Lock {
  private int turn;
  private boolean busy = false;

  public void lock() {
    int me = ThreadID.get();
    while (turn != me) {
      while (busy) {
        turn = me;
      }
      busy = true;
    }
  }

  public void unlock() {
    busy = false;
  }
}
\end{lstlisting}
    \end{column}
    \begin{column}{0.50\textwidth}
      \begin{itemize}
      \item Does this protocol satisfy mutual exclusion?
      \item Is this protocol starvation-free?
      \item Is this protocol deadlock-free?
  \end{itemize}
    \end{column}
  \end{columns}
\end{frame}

\begin{frame}{Lock}
\begin{lstlisting}
class MyLock implements Lock {
  private int turn;
  private boolean busy = false;

  public void lock() {
/* S1 */    int me = ThreadID.get();
/* S2 */    while (turn != me) {
/* S3 */      while (busy) {
/* S4 */        turn = me;
              }
/* S5 */      busy = true;
            }
  }

  public void unlock() {
/* S6 */    busy = false;
  }
}
\end{lstlisting}
\end{frame}

\begin{frame}{TODO}
  
\end{frame}

\begin{frame}{TODO}
  
\end{frame}

\begin{frame}{TODO}
  
\end{frame}

\begin{frame}{TODO}
  
\end{frame}


\section{JCSP: MergeSort}

\begin{frame}{Outline}
  \tableofcontents[current]
\end{frame}

\begin{frame}{TODO}

\end{frame}

\begin{frame}{TODO}

\end{frame}

\begin{frame}{TODO}

\end{frame}

\begin{frame}{TODO}

\end{frame}

\section{JCSP: Dining Philosophers}

\begin{frame}{Outline}
  \tableofcontents[current]
\end{frame}

\begin{frame}{TODO}
http://en.wikipedia.org/wiki/File:Dining_philosophers.png
\end{frame}

\begin{frame}{TODO}

\end{frame}

\begin{frame}{TODO}

\end{frame}

\begin{frame}{TODO}

\end{frame}


\section*{Outro}

\begin{frame}{Summary}
  \begin{itemize}
  \item TODO
\end{itemize}
\end{frame}

\end{document}
