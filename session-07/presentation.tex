%==============================================================================
% presentation.tex
%==============================================================================


%==============================================================================
% Configuration
%==============================================================================

% Internationalisation
\usepackage[utf8]{inputenc}
\usepackage[T1]{fontenc}
% \usepackage[ngerman]{babel}

% Different packages
\usepackage{url}
\usepackage{color,listings,paralist}
\usepackage{enumerate}
\usepackage{tabularx}
\usepackage{alltt}

% Use default Acrobat reader fonts
\usepackage{mathpazo}

% Use CM fonts (increases document size)
\usepackage{ae}

% Use images
\usepackage{graphicx}

% Configure beamer
\usetheme[secheader]{Ikhono}
\usefonttheme[onlylarge]{structurebold}
\setbeamertemplate{navigation symbols}{}

% Variables
\providecommand{\Title}{Parallel Programming}
\providecommand{\Subtitle}{Recitation Session 7}
\providecommand{\Author}{Thomas Weibel <weibelt@ethz.ch>}
\providecommand{\Institute}{Laboratory for Software Technology, \\
  Swiss Federal Institute of Technology Z\"urich}
\providecommand{\Date}{April 22, 2010}

% PDF settings
\hypersetup{
  pdftitle={\Title, \Subtitle},
  pdfauthor={\Author},
  pdfsubject={\Institute},
  pdfkeywords={parallel programming} 
}

% Titlepage
\title{\Title}
\subtitle{\Subtitle}
\author{\Author}
\institute{\Institute}
\date{\Date}

% Listings
\lstdefinestyle{Default}{
  language=Java,
  tabsize=2,
  mathescape=true,
  inputencoding=utf8,
  showstringspaces=false,
  fontadjust=true,
  basicstyle=\ttfamily,
  keywordstyle=\color{blue}\bfseries,
}
\lstset{style=Default}


%==============================================================================
% Document
%==============================================================================

\begin{document}


% Titlepage
\begin{frame}[plain]
  \titlepage
\end{frame}


\section*{Introduction}

\begin{frame}{Executive Summary}
  \begin{itemize}
  \item TODO
  \end{itemize}
\end{frame}


\section{Determining when a Thread has finished}

\begin{frame}{Outline}
  \tableofcontents[current]
\end{frame}

\begin{frame}[fragile]{\lstinline{isAlive()}}
\begin{lstlisting}
// Create and start a thread 
Thread thread = new MyThread(); 
thread.start(); 

// Check if the thread has finished 
// in a non-blocking way 
if (thread.isAlive()) { 
  // Thread has not finished 
} else { 
  // Finished 
} 
\end{lstlisting}
\end{frame}

\begin{frame}[fragile]{\lstinline{join(delayMillis)}}
\begin{lstlisting}
// Wait for the thread to finish but don't 
// wait longer than a specified time 
long delayMillis = 5000; // 5 seconds 
try { 
  thread.join(delayMillis); 
  if (thread.isAlive()) { 
    // Timeout occurred, 
    // thread has not finished 
  } else { 
    // Finished 
  } 
} catch (InterruptedException e) { 
  // Thread was interrupted 
} 
\end{lstlisting}
\end{frame}

\begin{frame}[fragile]{\lstinline{join()}}
\begin{lstlisting}
// Wait indefinitely for the thread to finish 
try { 
  thread.join(); 
  // Finished 
} catch (InterruptedException e) { 
  // Thread was interrupted 
} 
\end{lstlisting}
\end{frame}


\section{Volatile}

\begin{frame}{Outline}
  \tableofcontents[current]
\end{frame}

\begin{frame}{TODO}
  \begin{itemize}
  \item TODO
  \end{itemize}
\end{frame}


\section{Last Assignment}

\begin{frame}{Outline}
  \tableofcontents[current]
\end{frame}

\begin{frame}{TODO}
  \begin{itemize}
  \item TODO
  \end{itemize}
\end{frame}


\section{TODO}

\begin{frame}{Outline}
  \tableofcontents[current]
\end{frame}

\begin{frame}{TODO}
  \begin{itemize}
  \item TODO
  \end{itemize}
\end{frame}


\section*{Outro}

\begin{frame}{Summary}
  \begin{itemize}
  \item TODO
  \end{itemize}
\end{frame}

\end{document}
