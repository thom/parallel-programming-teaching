%==============================================================================
% presentation.tex
%==============================================================================


%==============================================================================
% Configuration
%==============================================================================

% Internationalisation
\usepackage[utf8]{inputenc}
\usepackage[T1]{fontenc}
% \usepackage[ngerman]{babel}

% Different packages
\usepackage{url}
\usepackage{color,listings,paralist}
\usepackage{enumerate}
\usepackage{tabularx}
\usepackage{alltt}

% Use default Acrobat reader fonts
\usepackage{mathpazo}

% Use CM fonts (increases document size)
\usepackage{ae}

% Use images
\usepackage{graphicx}

% Configure beamer
\usetheme[secheader]{Ikhono}
\usefonttheme[onlylarge]{structurebold}
\setbeamertemplate{navigation symbols}{}

% Variables
\providecommand{\Title}{Parallel Programming}
\providecommand{\Subtitle}{Recitation Session 2}
\providecommand{\Author}{Thomas Weibel <weibelt@ethz.ch>}
\providecommand{\Institute}{Laboratory for Software Technology, \\
  Swiss Federal Institute of Technology Z\"urich}
\providecommand{\Date}{\today}

% PDF settings
\hypersetup{
  pdftitle={\Title, \Subtitle},
  pdfauthor={\Author},
  pdfsubject={\Institute},
  pdfkeywords={parallel programming} 
}

% Titlepage
\title{\Title}
\subtitle{\Subtitle}
\author{\Author}
\institute{\Institute}
\date{\Date}

% Listings
\lstdefinestyle{Default}{
  language=Java,
  tabsize=2,
  mathescape=true,
  inputencoding=utf8,
  showstringspaces=false,
  fontadjust=true,
  basicstyle=\ttfamily,
  keywordstyle=\color{blue}\bfseries,
}
\lstset{style=Default}


%==============================================================================
% Document
%==============================================================================

\begin{document}


% Titlepage
\begin{frame}[plain]
  \titlepage
\end{frame}

\note{
}


\section*{Introduction}

\begin{frame}{Executive Summary}
  \begin{itemize}
  \item TODO
  \end{itemize}
\end{frame}


\section{Last Assignment}

\begin{frame}{Outline}
  \tableofcontents[current]
\end{frame}

\begin{frame}{Solution}
  \begin{itemize}
  \item TODO
  \end{itemize}
\end{frame}


\section{Foobar}

\begin{frame}{Outline}
  \tableofcontents[current]
\end{frame}

\begin{frame}{Foobar}
  \begin{itemize}
  \item TODO
  \end{itemize}
\end{frame}

\begin{frame}[fragile]{Code}
  \begin{lstlisting}
class HelloWorld {

  public static void main(String[] args) {
    System.out.println("Hello, World!");
  }

}
  \end{lstlisting}
\end{frame}


\section{New Assignment}

\begin{frame}{Outline}
  \tableofcontents[current]
\end{frame}

\begin{frame}{Foobar}
  \begin{itemize}
  \item TODO
  \end{itemize}
\end{frame}


\section*{Outro}

\begin{frame}{Summary}
  \begin{itemize}
  \item TODO
  \end{itemize}
\end{frame}

\end{document}
